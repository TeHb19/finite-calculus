\documentclass{article}
\newcommand{\ff}[1]{%
  ^{\underline{#1}}%
}
\usepackage{asymptote}
\usepackage{siunitx}
\usepackage{authblk}
\usepackage[utf8]{inputenc}
\usepackage[T2A]{fontenc}
\usepackage[russian]{babel}
\usepackage{amsfonts}
\usepackage{amsmath}
\DeclareRobustCommand{\stirling}{\genfrac\{\}{0pt}{}}

\author{David Gleich\thanks{%
	Перевод и редактировние статьи: Роговский Владимир}}
\title{Конечое исчисление:
	Руководство  для решения плохих сумм.}
\date{Октябрь 2021}
\begin{document}
\maketitle
\tableofcontents
\section{Как посчитать $\sum_{x=1}^{n} x^2$ ?}
Одна из задач, которую мы научимся решать, это как механически (т.е не думая слишком много) посчитать сумму
\begin{equation} \sum_{x=1}^n x^2.\end{equation}
Несмотря на то, что многие уже знают ответ на этот вопрос, наши поиски приведут к техникам, позволяющим легко посчитать плохие суммы, например
\begin{equation} \sum_{x=1}^{n} \sum_{y=1}^m {(x+y)^2} \end{equation}
и
\begin{equation} \sum_{x=0}^n {x2^x} \end{equation}
Поскольку мы будем использовать $\sum_{x=1}^n {x^2}$ как пример, сначала посмотрим на первые несколько значений этой функции.
\begin{center}
\begin{tabular}{c|c c c c c c}
{n} & 1 & 2 & 3 & 4 & 5 & 6 \\ \hline
$\sum_{x=1}^n {x^2}$ & 1 & 5 & 14 & 30 & 55 & 91 
\end{tabular}
\end{center}
Сообразительный читатель, наверное, уже понял, что мы как-то сделаем связь с матанализом. Поэтому давайте посмотрим, что будет если мы просто ``притворимся'', что это был интеграл из матанализа. Заменив $\sum$ на $\int$, мы получаем
$$\sum_{x=1}^n x^2 \stackrel{?}{=} \int_1^n x^2 dx$$
Используя матанализ, мы сразу считаем интеграл
$$\sum_{x=1}^n x^2 \stackrel{?}{=} \frac{n^3}{3}-\frac{1}{3},$$
К сожалению, подставляя $n=2$, видно, что это неверно.
$$\sum_{x=1}^2 x^2 \stackrel{?}{=}  \frac{2^3}{3}-\frac{1}{3} = \frac{7}{3} \neq 5.$$
Графически, мы видим, что пошло не так. \\

\begin{center}
\begin{asy}
size(6cm);
draw((0, 0) -- (2, 0), EndArrow(10));
draw((0, 0) -- (0, 4), EndArrow(10));
label("$0$", (0,0), S);
label("$1$", (1,0), S);
label("$2$", (2,0), S);
label("$1$", (0,1), W);
label("$2$", (0,2), W);
label("$3$", (0,3), W);
label("$4$", (0,4), W);
draw((0, 1) --(1, 1));
draw((1, 1) --(1, 4));
draw((1, 4) --(2, 4));
draw((0, 0)..(1, 1)..(2,4));
\end{asy}
\end{center}

Нам нужна площадь под ступенчатой линией, а не площадь под кривой.
Иронично, что мы можем легко посчитать площадь под кривой, используя матанализ, но есть проблемы с определением более простой площади под ступенчатой линией. Вкратце, наша цель - найти конечный аналог матанализа, чтобы конечная сумма
$$ \sum_{x=1}^2 x^2 $$
была не сложнее, чем "бесконечная" сумма
$$\int_1^2 x^2 dx.$$ 
Кроме конечного исчисления, другой способ посчитать значение $\sum_{x=1}^n x^2$, это подсмотреть в книжке или попытаться угадать ответ. Посмотрев на 21 страницу 
``CRC Standart Mathematical Tables and Formuale'', мы найдём
$$\sum_{x=1}^n x^2 = \frac{n(n+1)(2n+1)}{6}$$
Не думаю, что я бы смог такое угадать.
\section{Вычислительная стоимость метода Гаусса}
Теперь немного отклонимся от темы. Сейчас я бы хотел показать, как наш пример суммы встречается на практике. Не в уравнении, которое появилось из ниоткуда,  а во время вычислительного анализа метода Гаусса - фундаментального алгоритма в линейной алгебре. Метод Гаусса получает матрицу $A$ размером $m*m$ и расчитывает нижнию треугольную матрицу $L$ и верхнюю треугольную матрицу $U$, чтобы $A=LU$.
\par\noindent\rule{\textwidth}{0.4pt}
\textbf{Алгоритм 1} Метод Гаусса
\par\noindent\rule{\textwidth}{0.4pt}
\textbf{Ввод:} $A  \in \mathbb{R}^{m \times m}$

$U = A, L = I$

\textbf{для} $k=1$ до $m-1$ сделать

\hphantom{    } \textbf{для} $j=k+1$ до $m$ сделать

\hphantom{        } $l_{jk}=u_{jk}/u_{kk}$

\hphantom{        } $u_{j,k:m}=u_{j,k:m}-l_{jk}u_{k,k:m}$

\hphantom{    }\textbf{закончить}

\textbf{закончить}

\par\noindent\rule{\textwidth}{0.4pt}

Тут я не буду выводить метод Гаусса, но просто дам алгоритм и буду анализировать его вычислительную стоимость.
Вычислительная стоимость алгоритма - это количество сложений, вычитаний, умножений и делений, выполненных в алгоритме. Чтобы анализировать эту стоимость, мы посмотрим сколько работы выполнено на каждом шаге внешнего цикла. В случае, когда $k=1$ , тогда $j$ принимает значения от 2 до $m$. Во внутреннем цикле, мы совершаем 1 деление, $m$ умножений и $m$ вычитаний. Поскольку этот цикл выполняется $m-1$ раз, то
$$2m(m-1)+(m-1)=2m^2-1 \leq 2m^2$$
 работы выполнено на первом шаге. На втором $j$ принимает значения от 3 до $m$. Во внутреннем цикле, мы совершаем 1 деление, $m-1$ умножение и $m-1$ вычитание, и в итоге
$$2(m-2)(m-1)+(m-2) \leq 2(m-1)^2.$$
Это продолжается также для оставшихся шагов.
Значит, общее количество шагов для метода Гаусса меньше, чем
$$ 2m^2+2(m-1)^2+2(m-2)^2+\dots+1=\sum_{x=1}^m {2x^2} = 2\sum_{x=1}^m x^2 .$$
Поскольку мы подсмотрели значение этой суммы ранее, мы знаем, что метод Гаусса использует менее
$$2\sum_{x=1}^n x^2 =2\frac{m(m+1)(2m+1)}{6}=\frac{2}{3}m^3+m^2+\frac{1}{3}$$
сложений,вычитаний,умножений и делений. Теперь давайте начнём определять, как мы сами можем посчитать $\sum_{x=1}^n x^2$.

\section{Конечное исчисление}
До того, как мы начнём выводить конечное исчисление, мне наверное нужно сначала объяснить, что это. По аналогии с матанализом, мы ищем ``закрытую форму'' для суммы вида
$$\sum_{x=a}^b f(x)$$
Под закрытой формой, мы подразумеваем какой-то набор фундмаментальных операций, например, сложение, умножение, возведение в степень и даже факториал. В матанализе, нам нужно считать площадь под графиком функции, а в конечном исчислении - площадь под последовательностью. Мы называем это конечным исчислением, потому что каждая сумма состоит из конечного (а не бесконечного) количества членов. Один способ думать о конечном исчислении - это матанализ на множестве целых чисел, а не действительных.
В матанализе мы используем производную и первообразную вместе с фундаментальной теоремой матанализа, чтобы записать закрытую форму выражения
$$\int_a^b f(x) dx = F(b)-F(a),$$
где
$$\frac{d}{dx} F(x)=f(x).$$
Наша первая цель в конечном исчислении это получении фундаментальной теоремы конечного исчисления похожей формы.
\subsection{Дискретная производная}
До того, как мы можем надеяться получить фундаметнальную теорему конечного исчисления, наша первая цель - это получить понятие ``производной''. Вспомним из матанализа, что
$$\frac{d}{dx} f(x) = \lim_{h \to 0} \frac{f(x+h)-f(x)}{h}$$
Поскольку мы работаем над целыми числами, самое близкое, что мы можем получить к $0$ это $1$ (если же не доходить до самого $0$). Тогда, дискретная производная - это
$$ \Delta f(x)=f(x+1)-f(x).$$
\textbf{Определение (Дискретная производная).} \textit{Дискретная производная от $f(x)$ определена как}
$$ \Delta f(x)=f(x+1)-f(x).$$
Что мы можем делать с нашей новой производной? Из матанализа мы знаем, что легко взять производную от степенных функций $f(x)=x^m$.
$$\frac{d}{dx}x^m=mx^{m-1}.$$
Надеемся, что мы найдем настолько же простую производную и для конечных степеней.
$$\Delta x^m = (x+1)^m-x^m \stackrel{?}{=} mx^{m-1}.$$
Быстрая проверка показывает, что
$$\Delta x=(x+1)-x=1,$$
К сожалению, эта простая формула не работает для $x^2$.
$$\Delta x^2=(x+1)^2-x^2=2x+1 \neq 2x.$$
Наша производная для $x^2$ отличается только на $1$. Есть ли какой-то лёгкий способ решить эту проблему? Если мы перепишем нашу предыдущую производную как
$$\Delta (x^2)-1=2x,$$
мы подошли ближе к ответу. Теперь, мы вносим $-1$ в нашу производную, замечая, что $\Delta (x^2-x)=2x+1-1=2x.$ Мы можем переписать $x^2-x$ как $x(x-1)$. Значит мы получили, что
$$\Delta (x(x-1))=2x.$$
Я оставлю читателю проверить, что
$$\Delta (x(x-1)(x-2))=3x(x-1).$$
Кажется, что мы нашли новый тип степеней для наших дискретных производных! Такие степени известны как ``убывающие степени.''

\textbf{Определение (Убывающая степень).} \textit{Выражение $x$ в убывающей $m$ записывается $x\ff{m}$}. Значение
$$x\ff{m}=x(x-1)(x-2) \dots (x-(m-1)).$$
Используя убывающие степени, мы докажем, что они являются аналогами обычных степеней в конечном исчислении.

\textbf{Теорема 3.1} \textit{Дискретная производная убывающей степени -  это показатель, умноженный на следущую убывающую степень. То есть,}
$$\Delta x\ff{m}=mx\ff{m-1}.$$
\textit{Доказательство.} Доказательство - это просто алгебра.
\begin{eqnarray*}
\Delta x\ff{m} &=&(x+1)\ff{m}-x\ff{m} \\
& = & (x+1)x(x-1) \dots (x-m+2)-x(x-1) \dots (x-m+2)(x-m+1) \\
& = & (x+1-x+m-1)x(x-1) \dots (x-m+2) \\
& = & mx\ff{m-1}. \\
& \square
\end{eqnarray*}
Теперь докажем ещё пару полезных теорем.

\textbf{Теорема 3.2} \textit{Дискретная производная суммы двух функций - это сумма дискретных производных этих функций.}
$$\Delta (f(x)+g(x))=\Delta f(x) + \Delta g(x)$$
\textit{Доказательство.} Доказательство прямо из определения.
$$\Delta (f(x)+g(x))=f(x+1)+g(x+1)-f(x)-g(x).$$
Переставляя члены, мы получаем
$$\Delta (f(x)+g(x))=\Delta f(x) + \Delta g(x)$$
$\square$

\textbf{Теорема 3.3} \textit{Дискретная производная постоянной, умноженной на функцию, - это постоянная, умноженная на дискретную производную функции.}
$$\Delta(cf(x))=c\Delta f(x).$$
\textit{Доказательство.} Просто выносим постоянную из определения дискретной производной.
$\square$
\subsection{Неопределённая сумма и дискретная первообразная.}
Немного понимая, что делает наша дискретная производная, перейдём к дискретному интегрированию. Сначала введём обозначения.

\textbf{Определение (Дискретная первообразная.)} \textit{Функция $f(x)$ со свойством $\Delta f(x)=g(x)$ называется первообразной функции $g$. Мы обозначаем класс таких функций, как неопределённая сумма $g(x)$,}
$$\sum g(x) \delta x =f(x)+C,$$
\textit{где C - это произвольная постоянная. $\sum g(x) \delta x$ также называется неопределённой суммой $g(x)$ .}
Дискретная первообразная соответствует первообразной или неопределённому интегралу из матанализа.
$$\int g(x) dx=f(x)+C$$
$$\sum g(x) \delta x=f(x)+C$$
Пока что мы просто ввели определения. Мы еще не посчитали ни одной первообразной. Но мы можем легко это сделать. Вспомним, что
$$\Delta x\ff{m}=mx\ff{m-1}.$$
Используя этот факт, вместе с Теоремами 3.2,3.3, позволяет нам показать, что
$$\sum x\ff{m} \delta x=\frac{x\ff{m+1}}{m+1}+C.$$
Теперь, давайте поработаем над фундаментальной теоремой конечного исчисления. Сначала, нам нужно дать определение дискретому определённому интегралу или определённой сумме.

\textbf{Определение (Дискретный определённый интеграл).} \textit{Пусть, $\Delta f(x)=g(x).$ Тогда}
$$\sum_a^b g(x) \delta x=f(b)-f(a).$$
С дискретным определённым интегралом теорему, которую мы хотели бы получить (по аналогии с матанализом), это
$$\sum_a^b g(x) \delta x \stackrel{?}{=} \sum_{x=a}^b g(x).$$
К сожалению, быстрая подстановка показывает, что это неверно.
$$\sum_1^5 x \delta x=\frac{5\ff{2}}{2}-\frac{1\ff{2}}{2}=\frac{(5)(4)}{2}=10$$
Но $\sum_{x=1}^5 x=15$, значит теорема не верна. Но мы очень близки. Заметим, что
$$\sum_1^5 x \delta x=10=\sum_{x=1}^4 x.$$
Эта новая формула даёт нам фундаментальную теорему конечного исчисления.

\textbf{Теорема 3.4.} \textit{Фундаментальная теорема конечного исчисления - это}
$$\sum_a^b g(x) \delta x=\sum_{x=a}^{b-1} g(x).$$
\textit{Доказательство.} Опять же, доказательство это просто алгебра. Пусть, $\Delta f(x)=f(x+1)-f(x)=g(x).$
\begin{eqnarray*}
\sum_{x=a}^{b-1} g(x) & = & \sum_{x=a}^{b-1} f(x+1)-f(x) \\
& = & f(a+1)-f(a)+f(a+2)-f(a+1)+\dots \\
&  &  f(b)-f(b-1) \\
& = & f(b)-f(a),
\end{eqnarray*}
и $f(b)-f(a)=\sum_a^b g(x) \delta x$ по определению.
$\square$
\subsection{Полезные теоремы конечного исчисления}
Теперь, когда у нас есть наша фундаментальная теорема, этот раздел - это просто набор теорем, чтобы сделать конечгое исчисление полезным.
Одна из самых полезных функций в матанализе - $f(x)=e^x$. У этой специальной функции есть свойства
$$\frac{d}{dx} e^x=e^x$$
и
$$\int e^x dx=e^x+C.$$
Наша первая теорема это то, что есть аналогичная функция в конечном исчислении - функция, являющаяся своей же производной. Чтобы найти её, давайте ``округлим'' $e$. Если мы это правильно сделаем, то аналогом $e$ должно быть 2 или 3, да? Давайте посмотрим, какая подходит.
$$\Delta (2^x)=2^{x+1}-2^x=2*2^x-2^x=(2-1)2^x=2^x.$$
$$\Delta (3^x)=3^{x+1}-3^x=3*3^x-3^x=(3-1)3^x=2*3^x.$$
Значит, $e$ соответствует число 2.

\textbf{Теорема 3.5.} \textit{Функция $2^x$ удовлетворяет}
$$\Delta (2^x)=2^x$$
и
$$\sum 2^X \delta x=2^x+C.$$
Давайте посчитаем общую производную для экспоненты.
$$\Delta (c^x)=c^{x+1}-c^x=(c-1)c^x.$$
Поскольку, $c$ - это постоянная в этом выражении, мы можем сразу посчитать и первообразную.
$$\sum c^x \delta x=\frac{c^x}{c-1}+C.$$
Одна из важных теорем матанализа - это формула для $\frac{d}{dx}(u(x)v(x)).$ Давайте найдём соответсвующую формулу в конечном исчислении.
\begin{eqnarray*}
\Delta (u(x)v(x)) &=& u(x+1)v(x+1)-u(x)v(x) \\
& = & u(x+1)v(x+1)-u(x)v(x+1)+u(x+1)-u(x)v(x) \\
& = &  v(x+1)\delta u(x+1)+u(x)\delta v(x).
\end{eqnarray*}
Теперь мы можем использовать эту производную, чтобы написать формулу дискретного интегрирования по частям.

\textbf{Теорема 3.6.}
$$\sum u(x)\Delta v(x) \delta x=u(x)v(x)-\sum v(x+1)\Delta u(x) \delta x.$$ 
\textit{Доказательство.} Если мы возьмём первообразную обеих частей у
$$\Delta (u(x)v(x))=u(x)\Delta v(x)+v(x+1)\Delta u(x),$$
мы получим
$$u(x)v(x)=\sum u(x)\Delta v(x) \delta x + \sum v(x+1) \Delta u(x) \delta x.$$
Переставляя члены, мы получаем теорему.
$\square$

Есть ещё три факта, на которых я бы хотел закончить. Во-первых, давайте посмотрим на производную и первообразную некоторых комбинаторних функций.Во-вторых, работают ли наши формулы для убывающих степеней и их производных и для отрицательных степеней? И в-третьих, что насчёт $x\ff{-1}$?
Биномиальные коэффиценты часто появляются в комбинаторике. Давайте посчитаем $\Delta \binom{x}{k}.$ Если мы используем хорошо известную формулу $\binom{n}{k}=\binom{n-1}{k-1}+\binom{n-1}{k}$, мы можем легко показать, что
$$\Delta \binom{x}{k}=\binom{x+1}{k}-\binom{x}{k}=\binom{x}{k-1}.$$
Мы можем это так переписать:
$$\sum \binom{x}{k} \delta x=\binom{x}{k+1}+C.$$
Поскольку убывающая степень только определена для положительных показателей, вам, возможно, было интересно про отрицательные убывающие степени. Давайте теперь с ними разберёмся. Мы можем вывести определение, посмотрев на закономерность в убывающих степеней.
\begin{eqnarray*}
x\ff{3} & = & x(x-1)(x-2) \\
x\ff{2} & = & x(x-1) \\
x\ff{1} & = & x \\
x\ff{0} & = & 1
\end{eqnarray*}
Чтобы перейти  с $x\ff{3}$ на $x\ff{2}$, мы делим на $x-2$. Чтобы перейти с $x\ff{2}$ на $x\ff{1}$, мы делим на $x-1$. Продолжая эту закономерность, мы получаем
\begin{eqnarray*}
x\ff{-1} & = & x\ff{0}/(x+1)=\frac{1}{x+1} \\
x\ff{-2} & = & x\ff{-1}/(x+2)=\frac{1}{(x+1)(x+2)}
\end{eqnarray*}

\textbf{Определение (Отрицательная убывающая степень).}
$$x\ff{-m}=\frac{1}{(x+1)(x+2)\dots (x+m)}$$
Давайте проверим наше правило для производных убывающих степеней.
\begin{eqnarray*}
\Delta (x\ff{-m}) & = & \frac{1}{(x+2)(x+3)\dots (x+m+1)}-\frac{1}{(x+1)(x+2)\dots (x+m)} \\
& = & \frac{(x+1)-(x+m+1)}{(x+1)(x+2)\dots (x+m+1)} \\
& = & -mx\ff{-m-1}
\end{eqnarray*}
Удивительно, но наша формула работает! Из этого факта мы получаем, что наша формула для дискретной первообразной работает и для отрицательных степеней. Ну, она работает, если $m \neq -1$. В этом случае мы бы получили $\sum x\ff{-1}=\frac{x\ff{0}}{0}$, а это проблема. Вспомним из матанализа, что нам нужна была функция $ln(x)$ для интеграции $\frac{1}{x}$. Нет лёгкого способа интуитивно получить правильную функцию.

\textbf{Теорема 3.7.} \textit{Пусть,}
$$H_x=\frac{1}{1}+\frac{1}{2}+\dots+\frac{1}{x}$$
\textit{для целых $x$. Функция $H_x$ это первообразная от $\frac{1}{X}$.}

\textit{Доказательство.} 
$$\Delta H_x=\frac{1}{1}+\frac{1}{2}+\dots+\frac{1}{x+1}-\frac{1}{1}-\frac{1}{2}-\dots-\frac{1}{x}=\frac{1}{x+1}$$

$\square$

\section{Делаем конечное исчисление полезным: Стирлинг и его числа.}
Если вы дочитали до этого места, то наверное думаете: ``Хорошо, ты показал все эти хорошие факты про конечное исчисление. Но ты так и не показал, как решить
$$\sum_{x=1}^n x^2$$
как ты обещал в самом начале.'' Теперь этим и займёмся.

\begin{center}
\begin{tabular}{c c c}
$f(x)$       & $\Delta f(x)$     &$\sum f(x) \delta x$ \\ \hline
$x\ff{m}$     & $mx\ff{m-1}$     & $\frac{x\ff{m+1}}{m+1}$ \\
$x\ff{-1}$     &$-x\ff{-2}$     &$H_x$ \\
$2^x$    &$2^x$     &$2^x$\\
$c^x$     & $(c-1)c^x$    &$\frac{c^x}{c-1}$ \\
$\binom{x}{m}$     &$\binom{x}{m-1}$      &$\binom{x}{m+1}$ \\
$u(x)+v(x)$     & $\Delta u(x) + \Delta v(x)$     & $\sum u(x) \delta x + \sum v(x) \delta x$ \\
$u(x)v(x)$     &$u(x)\Delta v(x)+v(x+1)\Delta u(x)$     & \\
$u(x)\Delta v(x)$     &     & $u(x)v(x)-\sum v(x+1)\Delta u(x) \delta x$
\end{tabular}

Таблица 1: Тут множество полезных теорем о конечном исчислении. Они сделают расчёты плохих сумм лёгкими(или хотя бы выносимыми). Все из дискретных первообразных написаны без константы ради краткости.
\end{center}
Чтобы работать с обычными степенями, нужно сначала найти способ перейти между $x^m$ и $x\ff{m}$, чтобы мы могли использовать наши теоремы интеграции. Давайте посмотрим на все превращения, которые можем сделать сами.
$$x^0=x\ff{0}$$
$$x^1=x\ff{1}$$
$$x^2=x\ff{2}+x\ff{1}$$
$$x^3=???$$
Мы видим, что первые несколько степеней легко преобразовать, но более высокие степени не настолько очевидны. Давайте поработаем с
$$x^3=ax\ff{3}+bx\ff{2}+c\ff{1}$$
чтобы получить формулу для $a$,$b$ и $c$.
\begin{eqnarray*}
ax\ff{3}+bx\ff{2}+cx\ff{1} &=& ax(x-1)(x-2)+bx(x-1)+cx \\
& = & ax^3-3ax^2+2ax+bx^2-bx+cx \\
& = & ax^3+(b-3a)x^2+(2a-b+c)x
\end{eqnarray*}
Если мы хотим $ax\ff{3}+bx\ff{2}+cx\ff{1}=x^3,$
то нам нужно $a=1$, $b-3a=0$, $2a-b+c=0$.
Или иначе $a=1$, $b=3$ и $c=1$. Значит, мы имеем
$$x^3=x\ff{3}+3x\ff{2}+x\ff{1}$$
Это много работы просто для $x^3$. Я не буду пытаться это сделать для $x^4$, но если вы хотите, то можете попробовать. Вместо этого, остальную часть этого раздела я посвящу одной теореме.

\textbf{Теорема 4.1.} \textit{Мы можем переходить между степенями и убывающими степенями, используя эту формулу.}
$$x^m=\sum_{k=0}^m \stirling{m}{k} x\ff{k},$$
\textit{где $\stirling{m}{k}$ - число Стирлинга второго рода.}

Ладно, эту теорему сложно сразу переварить. До того, как мы начнём её доказывать, давайте разберём её на части. Теорема гласит, что мы можем переходить между степенями и убывающими степенями, если мы можем посчитать эти числа Стирлинга (до них мы скоро дойдём). Значит, если мы хотим выразить $x^4$ через убывающие степени, эта теорема говорит нам ответ:
$$x^4=\stirling{4}{0}x\ff{0}+\stirling{4}{1}x\ff{1}+\stirling{4}{2}x\ff{2}+\stirling{4}{3}x\ff{3}+\stirling{4}{4}x\ff{4}.$$
Теперь, давайте про эти числа Стирлинга.
\subsection{Числа Стирлинга (Второго рода)}
Если вы их раньше не видели, вам, наверное, стало интересно про эти числа Стирлинга. Для начала, я дам их определение.

\textbf{Определение (Числа Стирлинга).}  \textit{Числа Стирлинга второго рода, обозначаются}
$$\stirling{n}{k},$$
\textit{это количество способов разбить $n$ различных предметов на $k$ непустых множеств.}
Наверное, это определение не сильно помогло. Надеюсь, что работа с ними поможет. Сначала давайте попробуем посчитать $\stirling{1}{1}$. Это количество способов разбить 1 предмет на 1 непустое множество. Думаю, что очевидно, что есть только один способ это сделать. Значит, $\stirling{1}{1}=1$. А что насчёт $\stirling{1}{0}$, или количество разбить 1 предмет на 0 непустых множеств? Поскольку, каждое множество должно быть непустым, есть 0 способов это сделать, значит $\stirling{1}{0}=0$. Мы можем это обобщить и показать, что
$$\stirling{n}{0}=0$$
если $n>0$. А что насчет $\stirling{0}{0}$? Мы хотим разбить 0 предметов на 0 непустых множеств. Не должно ли и это равняться 0? Странно, но нет, $\stirling{0}{0}=1$. Если вам сложно принять этот факт логически, считайте, что это определение.
Ещё одни простые числа Стирлинга это $\stirling{n}{n}$. Эти числа являются количеством способов разбить $n$ предметов на $n$ непустых множеств. Поскольку количество предметов равно количеству множеств, в каждом множестве может быть только один предмет, значит $\stirling{n}{n}=1$.
Теперь давайте посмотрим на более сложное число Стирлинга, $\stirling{3}{2}$. Пусть наши 3 предмета это $\clubsuit,\diamondsuit,\heartsuit$. Когда мы делим эти 3 масти на 2 непустых множества,мы получаем следующие множества.
$$\{\clubsuit,\diamondsuit\},\{\heartsuit\}$$ 
$$\{\clubsuit,\heartsuit\},\{\diamondsuit\}$$
$$\{\heartsuit,\diamondsuit\},\{\clubsuit\}$$
Значит, $\stirling{3}{2}=3$, поскольку есть 3 способа разбить 3 предмета на 2 непустых множества.
Но мы всё равно не многое знаем про числа Стирлинга. Давайте докажем теорему, чтобы мы могли считать числа Стирлинга, зная предыдущие.

\textbf{Теорема 4.2.}
$$\stirling{n}{k}=\stirling{n-1}{k-1}+k\stirling{n-1}{k}$$
\textit{Доказательство.} Мы дадим комбинаторное доказательство этому. В этом доказательстве, мы посчитаем количество способов разбить $n$ предметов на $k$ не пустых множеств.
Очевидный способ это посчитать - это использовать наше определение чисел Стирлинга. По этому определению, количество способов будет равно $\stirling{n}{k}$.
Теперь давайте это посчитаем другим способом.Давайте выберем фиксированный предмет и назовём его``чудаком''. Тогда мы можем посчитать количество способов разбиения, посмотрев куда идёт чудак. Первый случай, когда чудак сам по себе в множнстве. В этом случае есть $\stirling{n-1}{k-1}$ разбить оставшиеся предметы. Второй случай, когда чудак с кем-то другим в множнстве. В этом случае, остальные предметы разделены на $k$  множеств. Тогда, общее количество способов разбиения в этом случае - $k*\stirling{n-1}{k}$.
Значит,
$$\stirling{n}{k}=\stirling{n-1}{k-1}+k\stirling{n-1}{k}.$$
$\square$
\subsection{Доказательство теоремы.}
Чтобы доказать теорему 4.1, сначала докажем полезную лемму, а потом уже много алгебры.

\textbf{Лемма} \textit{Полезная лемма}
$$x*x\ff{k}=x\ff{k+1}+kx\ff{k}$$.

\textit{Доказательство.}
\begin{eqnarray*}
x*x\ff{k} &=& x*x\ff{k}-k*x\ff{k}+k*x\ff{k} \\
&=& (x-k)*x\ff{k}+k*x\ff{k} \\
&=& x\ff{k+1}+k*x\ff{k} \\
\square
\end{eqnarray*}

Теперь, вспомним теорему 4.1.
$$x^m=\sum_{k=0}^m \stirling{m}{k} x\ff{k},$$

\textit{Доказательство.} Мы докажем это индукцией по $n$.
База индукции, $n=1$ тривиальна.
Предположение индукции это то, что $x^n=\sum_{k=0}^n \stirling{n}{k} x\ff{k}$ верно для всех $n \leq r$,
Переход. n=r+1
\begin{eqnarray*}
x^{r+1} &=& xx^r \\
&=& \sum_{k=0}^r \stirling{r}{k} xx\ff{k} \\
&=& \sum_{k=0}^r \stirling{r}{k} (x\ff{k+1}+kx\ff{k}) \\
&=& \sum_{k=1}^{r+1} \stirling{r}{k-1} x\ff{k} + \sum_{k=0}^r \stirling{r}{k} kx\ff{k} \\
&=& \stirling{r}{r}x\ff{r+1}+\sum_{k=1}^r(\stirling{r}{k-1}x\ff{k}+\stirling{r}{k}kx\ff{k})+\stirling{r}{0}x\ff{0} \\
&=& x\ff{r+1} + \sum_{k=0}^r \stirling{r+1}{k}x\ff{k} \\
&=& \sum_{k=0}^{r+1} \stirling{r+1}{k} x\ff{k} \\
\square
\end{eqnarray*}
Замечательно! Теперь мы можем переходить между $x^m$ и убывающими степенями!
\subsection{Вычисление чисел Стирлинга}
Мы можем вывести формулу для вычисления чисел Стирлинга, без рекурентного соотношения, полученного в теореме 4.2.

\textbf{Теорема 4.3.}
$$\stirling{n}{k}=\frac{1}{k!} \sum_{i=0}^k (-1)^i \binom{k}{i} (k-i)^n.$$

\textit{Доказательство.} Опять же, докажем это комбинаторно.
Тут мы будем считать количество суръекций между множеством из $n$ элементов $X$ и множеством из $k$ элементов $Y$.
Поскольку мы считаем суръекции, то каждому элементу из $Y$ соответствует непустое множество элементов из $X$. Значит количество таких суръекций это $k!\stirling{n}{k}$.
С другой стороны, мы можем посчитать это методом включения и исключения. Рассмотрим все возможные отображения из $X$ в $Y$ и отнимим количество отображений, где хотя бы одному элементу из $Y$ не соответствует ни один элемент из $X$. Используя метод велючения и исключения, мы получаем
$$\binom{k}{0}k^n-\binom{k}{1}(k-1)^n+\binom{k}{2}(k-2)^n-\dots=\sum_{i=0}^k (-1)^i \binom{k}{i} (k-i)^n.$$
Делим обе части на $k!$ для завершения доказательства. $\square$
\section{Множество примеров...}
В этом разделе, мы посмотрим на несколько примеров, чтобы показать полезность конченого исчисления.
\subsection{$\sum_{x=1}^n x^2$}
Поскольку я ещё не показал, как посчитать эту сумму, давайте это сделаем сейчас.
$$\sum_{x=1}^n x^2=\sum_1^{n+1}x^2 \delta x=\sum_1^{n+1} x\ff{2}+x\ff{1} \delta x=(n+1)\ff{3}/3+(n+1)\ff{2}/2-1\ff{3}/3-1\ff{2}/2=(n+1)\ff{3}/3+(n+1)\ff{2}/2.$$
\subsection{Двойная сумма}
Как и обещал в начале, давайте посчитаем $\sum_{x=1}^n \sum_{y=1}^n (x+y)^2$ используя конечное исчисление.
\begin{eqnarray*}
\sum_{x=1}^n \sum_{y=1}^n (x+y)^2 &=& \sum_{x=1}^{n+1} \sum_{y=1}^{n+1} (x+y)^2 \delta y \delta x=\sum_{x=1}^{n+1} \sum_{y=1}^{n+1} (x+y)\ff{2}+(x+y)\ff{1} \delta y \delta x \\
&=& \sum_1^{n+1} (x+n+1)\ff{3}/3+(x+n+1)\ff{2}/2-(x+1)\ff{3}/3-(x+1)\ff{2}/2 \delta x \\
&-& (2n+2)\ff{4}/12+(2n+2)\ff{3}/6-(n+2)\ff{4}/6-(n+2)\ff{3}/3
\end{eqnarray*}
Несмотря на то, что это верно, мы пропустили один шаг. Чему равняется $\sum(x+c)\ff{m} \delta x$? Мы предположили, что это равняется $\sum x\ff{m} \delta x$ без доказательства. Посмотрев на доказательство теоремы 3.1, это верно, ведь мы просто добавили постоянную к убывающей степени.
\subsection{Средняя длина команды}
Представьте, что вы глава программирования одного из марсоходов. Они принимают команды длиной до 10 бит. Их краткое содержание есть ниже.


\begin{center}
\begin{tabular}{c c}
0 & стоп \\
1 & взять фотографию \\
00 & двигаться вперёд \\
01 & двигаться назад \\
10 & двигаться влево \\
11 & двигаться вправо \\
\vdots & \vdots \\
001101 & взять фотографию зелёного пришельца \\
\vdots & \vdots \\
1111111111 & самоуничтожение
\end{tabular}
\end{center}

В этой таблице есть пара интересных вещей. Во-первых, 0 и 00 - это разные команды! Во-вторых, все возможные команды используются и нет избыточности в командах.
После того, как вы сделали эту систему команд, ваш босс звонит вам и хочет узнать среднюю длину команды. Поскольку NASA платит за межпланетную коммуникацию по битам, он сильно хочет знать сколько будет стоит ваша система.
Пока вы разговариваете с вашим боссом, вы делаете пару заметок.
Во-первых, существует ровно $2^x$ команд длиной $x$ бит.
Значит, общее количество команд это
$$\sum_{x=1}^{10} 2^x=2^{11}-2.$$
Во-вторых, общая длина всех команд длиной $x$ бит это $x2^x$. Ага! Теперь вы можете легко посчитать среднюю длину команды.
$$\frac{\sum_{x=1}^{10} x2^x}{\sum_{x=1}^{10} 2^x}.$$
Несмотря на то, что вы уже посчитали знаменатель, числитель чуть сложнее. Но, поскольку вы (очевидно) прочитали все части этой статьи, вы знаете про конечное исчислении и продожаете.
$$\sum_{x=1}^{10} x2^x=\sum_1^{11} x2^x \delta x.$$
Теперь вы вспоминаете маленькую теорему про ``дискретное интегрирование по частям'' (Теорема 3.6.)
$$\sum x2^x\delta x=x2^x-\sum 2^{x+1}\delta x=x2^x-2^{x+1}=2^x (x-2).$$
Тогда,
$$\sum_{x=1}^{10} x2^x=2^{11} (11-2)-2^1 (1-2)=9*2^{11}+2.$$
Тогда, средняя длина команды это
$$\frac{9*2^{11}+2}{2^{11}-2} \approx 9.00978 \approx 9.$$
После небольшой паузы в разговоре, вы с увереностью отвечаете: ``Сэр, средняя длина команды - 9 бит до двух знаков после запятой''
\end{document}